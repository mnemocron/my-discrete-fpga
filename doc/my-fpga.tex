\documentclass[twoside,final]{fhnwreport}       %[mode] = draft or final
                                        %{class} = fhnwreport, article, 
                                        %          report, book, beamer, standalone
%%---Main Packages-----------------------------------------------------------------------
\usepackage[english]{babel}	%Mul­tilin­gual sup­port for LaTeX
\usepackage[T1]{fontenc}				%Stan­dard pack­age for se­lect­ing font en­cod­ings
\usepackage[utf8]{inputenc}				%Ac­cept dif­fer­ent in­put en­cod­ings
\usepackage{lmodern}                    %The newer Font-Set
\usepackage{textcomp}					%LaTeX sup­port for the Text Com­pan­ion fonts
\usepackage{caption}					%Customising captions in floating environments
\usepackage{graphicx} 					%En­hanced sup­port for graph­ics
\usepackage{float}						%Im­proved in­ter­face for float­ing ob­jects
\usepackage{ifdraft}                    %Let you check if the doc is in draft mode
\usepackage{wrapfig}                    %Let wrap text around figures

%%---Useful Packages---------------------------------------------------------------------
\usepackage{color}						%Colour control for LaTeX documents
\usepackage[pdftex,dvipsnames]{xcolor}  %Driver-in­de­pen­dent color ex­ten­sions for LaTeX
\usepackage{csquotes}                   %Simpler quoting with \enquote{}
\usepackage{siunitx} 					%A com­pre­hen­sive (SI) units pack­age
\usepackage{listings}					%Type­set source code list­ings us­ing LaTeX
\usepackage[bottom]{footmisc}			%A range of foot­note op­tions
\usepackage{footnote}					%Im­prove on LaTeX's foot­note han­dling
\usepackage{verbatim}					%Reim­ple­men­ta­tion of and ex­ten­sions to LaTeX ver­ba­tim
\usepackage[textsize=footnotesize]{todonotes} %Mark­ing things to do in a LaTeX doc­u­ment
\usepackage{titling}					%Control over the typesetting of the \maketitle command
\usepackage{sectsty} 
\usepackage{fancyheadings}

%%---Tikz Packages-----------------------------------------------------------------------
\usepackage{standalone}
\usepackage{tikz}
\usepackage{circuitikz}
\usetikzlibrary{arrows}
\usetikzlibrary{calc}
\usetikzlibrary{intersections}

%%---Math Packages-----------------------------------------------------------------------
\usepackage{amsmath}					%AMS math­e­mat­i­cal fa­cil­i­ties for LaTeX
\usepackage{amssymb}					%Type­set­ting symbols (AMS style)
%\usepackage{amstext}
%\usepackage{amsfonts}
%\usepackage{breqn}
\usepackage{array}						%Ex­tend­ing the ar­ray and tab­u­lar en­vi­ron­ments
\usepackage{amsthm}					%Type­set­ting the­o­rems (AMS style)

%%---Table Packages----------------------------------------------------------------------
\usepackage{booktabs}
%\usepackage{vcell}
\usepackage{tabularx}					%Tab­u­lars with ad­justable-width columns
%\usepackage{longtable}
\usepackage{multirow}					%Create tab­u­lar cells span­ning mul­ti­ple rows
\usepackage{makecell}
\usepackage{multicol}					%In­ter­mix sin­gle and mul­ti­ple columns
\usepackage[normalem]{ulem}
\useunder{\uline}{\ul}{}

%%---PDF / Figure Packages---------------------------------------------------------------
\usepackage{pdfpages}					%In­clude PDF doc­u­ments in LaTeX
\usepackage{pdflscape}					%Make land­scape pages dis­play as land­scape
\usepackage{subfig}					    %Fig­ures di­vided into sub­fig­ures
\usepackage{geometry} 

%%---Other Packages----------------------------------------------------------------------
%\usepackage{xargs}                     %De­fine com­mands with many op­tional ar­gu­ments
%\usepackage{svg}

\lstset{
	aboveskip=1ex,
	backgroundcolor=\color{gray!25},
	basicstyle=\small\ttfamily,
	belowskip=1ex,
	breaklines=true,
	columns=fullflexible,
	framerule=0pt,
	framexrightmargin=0em,
	framexleftmargin=0em,
	numbers=left,
	numberstyle=\footnotesize\sffamily,
	tabsize=2
}

\lstdefinestyle{DOS}
{
	backgroundcolor=\color{black},
	basicstyle=\scriptsize\color{white}\ttfamily
}

%%---Custom Title Style------------------------------------------------------------------
%% package: sectsty
\definecolor{SecColor}{rgb}{0.09, 0.17, 0.38} % dark blue
% \definecolor{SecColor}{rgb}{0.75,0.22,0.16} % red
\definecolor{SubSubSecColor}{rgb}{0.11,0.21,0.59}
\definecolor{SubSecColor}{rgb}{0.15,0.23,0.45}

% \definecolor{URLColor}{rgb}{30, 55, 153}

%\allsectionsfont{\sffamily} % Sans Serif font for all headings
\sectionfont{\sffamily\huge\color{SecColor}}  % sets colour of sections
\subsectionfont{\sffamily\LARGE\color{SubSecColor}}
\subsubsectionfont{\sffamily\Large\color{SubSubSecColor}}
\paragraphfont{\sffamily\color{SecColor}}

%%---Bibliography------------------------------------------------------------------------
\usepackage[style=ieee,urldate=comp,backend=biber]{biblatex}
\addbibresource{literature/bibliography.bib}

%%---Main Settings-----------------------------------------------------------------------
\graphicspath{{./graphics/}}			%Defines the graphicspath
\geometry{twoside=true}				    %twoside=false disables the "bookstyle"
\setlength{\marginparwidth}{2cm}
\overfullrule=5em						%Creates a black rule if text goes over the margins => debugging

\usepackage{glossaries}
\makeglossaries


%%---User Definitions--------------------------------------------------------------------
%%Tabel-Definitions: (requires \usepackage{tabularx})
\newcolumntype{L}[1]{>{\raggedright\arraybackslash}p{#1}}    %column-width and alignment
\newcolumntype{C}[1]{>{\centering\arraybackslash}p{#1}}
\newcolumntype{R}[1]{>{\raggedleft\arraybackslash}p{#1}}

%%---Optional Package Settings-----------------------------------------------------------
%Listings-Settings: (requires \usepackage{listings}) => Example with Matlab Code
\definecolor{VitisGray}{rgb}{0.95,0.95,0.95}
\lstset{ 
	backgroundcolor=\color{VitisGray},   % choose the background color; you must add \usepackage{color} or \usepackage{xcolor}; should come as last argument
	basicstyle=\footnotesize\ttfamily,   % the size of the fonts that are used for the code
	breakatwhitespace=false,         % sets if automatic breaks should only happen at whitespace
	breaklines=false,                 % sets automatic line breaking
	captionpos=b,                    % sets the caption-position to bottom
	escapeinside={\%*}{*)},          % if you want to add LaTeX within your code
	firstnumber=1,                   % start line enumeration with line 1
	frame=single,	                   % adds a frame around the code
	keepspaces=true,                 % keeps spaces in text, useful for keeping indentation of code (possibly needs columns=flexible)
	keywordstyle=\color{blue},       % keyword style
	language=c,                 % the language of the code
	numbers=left,                    % where to put the line-numbers; possible values are (none, left, right)
	numbersep=5pt,                   % how far the line-numbers are from the code
	numberstyle=\color{darkgray}, % the style that is used for the line-numbers
	rulecolor=\color{black},         % if not set, the frame-color may be changed on line-breaks within not-black text (e.g. comments (green here))
	showspaces=false,                % show spaces everywhere adding particular underscores; it overrides 'showstringspaces'
	showstringspaces=false,          % underline spaces within strings only
	showtabs=false,                  % show tabs within strings adding particular underscores
	tabsize=2,	                   % sets default tabsize to 2 spaces
	title=\lstname                   % show the filename of files included with \lstinputlisting; also try caption instead of title
}

%\lstset{language=Matlab,%
%	captionpos=b,                    % sets the caption-position to bottom
%	basicstyle=\footnotesize\ttfamily,
%	breaklines=false,%
%	morekeywords={switch, case, otherwise},
%	keywordstyle=\color{Blue},%
%	tabsize=2,
%	%morekeywords=[2]{1}, keywordstyle=[2]{\color{black}},
%	identifierstyle=\color{Black},%
%	stringstyle=\color{Purple},
%	commentstyle=\color{Green},%
%	showstringspaces=false,%without this there will be a symbol in the places where there is a space
%	numbers=left,%
%	numberstyle={\tiny \color{black}},% size of the numbers
%	numbersep=9pt, % this defines how far the numbers are from the text
%	%emph=[1]{word1, word2,...},emphstyle=[1]\color{red}
%}

\lstdefinestyle{Vitis}{
	basicstyle=\ttfamily,
	language=C++,
	emphstyle={\color{blue}\bfseries},
%	captionpos=b,                    % sets the caption-position to bottom
%	morekeywords={u8, u16, u32},
	keywordstyle={\bfseries\color{Purple}},
	identifierstyle=\color{Black},%
	stringstyle=\color{blue},
	commentstyle=\color{Green},%
	frame=single,	                   % adds a frame around the code
	keepspaces=true,
	    % keyword highlighting
	classoffset=1, % starting new class
	otherkeywords={u8, u16, u32, u64},
	morekeywords={u8, u16, u32, u64},
	keywordstyle={\bfseries\color{blue}},
	classoffset=0,
}

%Hurenkinder und Schusterjungen verhindern (kein Scherz, Google es)
\clubpenalty10000
\widowpenalty10000
\displaywidowpenalty=10000	


%%---Costum Packages Bachelor Thesis-------------------------------------------
\usepackage{diagbox}
\usepackage{xurl}

\usepackage{hyperref}
\hypersetup{
    colorlinks=true,
    linkcolor=black,
    filecolor=black,       
    urlcolor=blue,
    citecolor=black,
}			                %loads all packages, definitions and settings
\addbibresource{literature/bibliography.bib}							
%\title{Reference Clock Spur Mitigation in RF-DAC using Digital Predistortion (DPD) via I/Q Modulation using the AD91xx RF-DAC Familiy}  		        %Project Title
\title{\sffamily The Worst FPGA Ever Built}  		        %Project Title
\author{Burkhardt Simon}      				    %Document Type => Technical Report, ...
\date{\today}          				   %Place and Date

\setcounter{tocdepth}{3} %%% Tabble of Content Depth


\begin{document}

%%---TITLEPAGE---------------------------------------------------------------------------------

% Das Titelblatt macht präzise Angaben zur Autorenschaft, nennt Auftraggeber und Betreuer und ist datiert. 
% Material: Titel aus der Aufgabenstellung, Auftragsfirma oder eigener Titel (Untertitel = Titel aus der
% Aufgabenstel-lung) 
% Adressaten: Auftraggeber, Fachcoachs, künftige Projektentwickler sowie alle Interessierten 
% Das Titelblatt ist immer ein gesondertes Blatt. Auf dem Titelblatt stehen: 
% Titel der Aufgabe, ev. Untertitel, Art der Arbeit (Fachbericht, Bachelor-Thesis), 
% Projektnummer (sofern vorhanden), Name des Auftraggebers, der Betreuer offizieller von den Dozenten gegebener oder 
% evtl. selbst kreierter Teamname (z.B. Team 3), Name der Verfasserinnen, der Verfasser, Klasse, Semester, in dem der 
% Bericht verfasst wurde, Studiengang, Namen der betreuenden Personen, Ort und Datum. Das Titelblatt soll informativ, 
% sachlich und nüchtern sein

\thispagestyle{empty}
%	\ohead{\includegraphics[scale=0.5]{Bilder/Logo_FHNW.jpg}}
	\begin{figure}
		\vspace*{-\topskip}\vspace*{-\headsep}
		%\includegraphics[scale=1]{fhnw_ht_e_10mm.jpg} %{graphics/fhnw_ht_logo_de.pdf}
		\hspace{1cm}
%		\includegraphics[scale=0.5]{swissuniversities_logo_schwarz.jpg}
	\end{figure}
	\begin{center}
		\vspace*{2.0cm}
		{\huge{\textbf{\thetitle}}}\\
		\vspace*{0.5cm}
		{\Large{\textbf{\sffamily Designing my own FPGA \\
					using Discrete Electronics \\
					on Circuit Boards }}}
		\vspace*{0.5cm}
		
%		{\large{A}} \\
%		\vspace*{0.5cm}
		
%		{\huge{\sffamily Master Thesis}}\\
%		\vspace*{0.5cm}
		
%		{\large{By}} \\
%		\vspace*{0.5cm}
		
		{\large{\sffamily Simon Burkhardt}} \\
		\vspace*{0.5cm}
		
%		{\large{Submitted in partial fulfillment of the requirements \\ 
%				for a degree of \\
%			Master of Science in Engineering}} \\
		%\vspace*{0.5cm}
		
		%\vspace*{0.5cm}
%		{\large{At}} \\
%		\vspace*{0.5cm}
%		{\large{University of Applied Sciences and Arts Northwestern Switzerland}} \\
%		\vspace*{0.5cm}
		
		%\Large{\selectlanguage{english}\today}
		\large{June, 2023}
		
	%	\begin{figure}[H]
	%		\centering
			%\includegraphics[width=0.5\linewidth]{graphics/title_teaser.pdf}
	%	\end{figure}
		
		\vfill
		
		\begin{normalsize}
			{\begin{tabbing}
%					\textbf{External Expert} \hspace{3cm}\=Dr. Jürg Stettbacher \\
%					\textbf{Supervisor} \>Prof. Michael Pichler \\
%					\textbf{} \>Prof. Karl Schenk \\
%					\textbf{Advisor} \>Prof. Dr. Christoph Wildfeuer \\
%					\textbf{Industrial Partner} \> AnaPico AG, 8152 Opfikon \\
			\end{tabbing}}
		\end{normalsize}
		%\vfill
	\end{center}
\clearpage
\thispagestyle{empty}
\begin{center}
This document was created with \LaTeX.\\
\end{center}
\clearpage
%%---ABSTRACT----------------------------------------------------------------------------
\selectlanguage{english}				%ngerman or english
\thispagestyle{empty}
\include{sections/0_0_0_Abstract}

\clearpage
\thispagestyle{empty}
\paragraph{Acknowledgements}\,\\

I would like to express my deepest appreciation to my professor and supervisor Michael Pichler for his invaluable incentive to understand the topic and for sparking my joy in FPGA engineering.
Additionally, this endeavor would not have been possible without the generous support from the Innosuisse Agency, who partially financed this research.

Special thanks goes to AnaPico AG for their funding and their valuable technical expertise. I am also grateful to my colleagues at FHNW for their expertise and moral support that impacted and inspired me.

\clearpage

%%---TABLE OF CONTENTS-------------------------------------------------------------------
\pagenumbering{Roman}		
\selectlanguage{english}				%ngerman or english
\tableofcontents
\clearpage

%%---TEXT--------------------------------------------------------------------------------
\pagenumbering{arabic}
\pagestyle{fancy}
\fancyhf{}
%\lhead{\sffamily\rightmark}
%\rhead{\thepage}
\fancyhead[LE,RO]{\thepage}
\fancyhead[RE,LO]{\sffamily\rightmark}
\cfoot{\sffamily \color{gray} PRELIMINARY }

% - Warum ist der Abschnitt vorhanden?
% - Was wird erklärt?
% - Kernaussage des Abschnittes?

% Was ist das (Teil-)Problem?
% ... dieses Teilbereichs/bzgl. Gesamtaufgabe
% Was kommt zuerst zur Sprach, was danach?
% Was wurde gemacht, um die Aufgabe zu lösen?
% ... Evalution, Berechnungen, Tests etc. (z.B. Grafiken)
% Was ist rausgekommen?
% ... die Resultate, die Entscheide, begründet
% Was schliessen Sie daraus?

% - Einbindung: Grafik ankündigen
% - Bildlegende: Grafik kennzeichnen
% - Erläuterung: Grafik interpretieren

%%%% Überprüfen Sie Folgendes:
% Einstiegssatz: 
% - Informiert er den Leser über die Teil-/Aufgabe? 
% - Sind die Anforderungen genannt?

% Innere Logik: 
% - Stimmt die Reihenfolge? à roter Faden
% - Sind die Arbeitsschritte logisch nacheinander?

% Erläuterung: 
% - Sind die grafischen Elemente:  angekündigt, genannt und kommentiert

%%%% Rechtschreibprüfung in Deutsch
% - Wortschatz, Synonyme: https://www.woxikon.de
% - https://wortliga.de/textanalyse/. 
% - Lesbarkeit (FREI): www.leichtlesbar.ch
% - Textprüfung: mentor.duden.de



% 1. Introduction
% 2. Literature Review
% 3. Data and Methodology
% 4. Results
% 5. Discussion

% reference to datasheet pages
% (\textit{Table 2-3 on p. 16} of PG203 \cite{PG203})

%\input{sections/acronyms}
\clearpage
\include{sections/Introduction}
	\include{sections/InspirationalFPGA}
	\subsection{Ben Eaters 8 Bit CPU}


	\subsection{The End Goal of this Project}

With the expected cost being way beyond 1'000 USD all while expecting a performance not even close to 20 year old microcontrollers, it is hard to justify this project.
I admit that designing such a machinery will not deliver any reasonable performance let alone create any innovation.
Yet, it will teach me (and hopefully you) about the inner workings of an FPGA and showcase all the difficulties that designers of modern FPGA chips face.

What I aim to achieve is to build an FPGA with enough complexity to run a few relevant circuits. Thereby teaching myself the basics of FPGA architecture while maintaining a functional, reconfigurable hardware. 
The entire FPGA should be able to run basic circuits such as a 4-bit counter or a 4-bit BCD to 7-segment decoder.
Once the project grows further, more specialized logic can be added including block RAM or even DSP slices.

Since I do not plan to open another Pandoras box of synthesis and routing software algorithms, the bitstream is required to be manually managed resulting in a size constraint of around <4kB.



\section{A Brief History of FPGAs}



\section{FPGA Design Concepts \& Practices}

	\include{sections/concept_CLB}
		\subsubsection{Look Up Table (LUT)}

The LUT is the implementation of a truth-table in hardware.
Any thinkable logic function with $A$ inputs and $O$ outputs can be realized using a $O\cdot 2^A$ LUT.

Modern implementations of LUTs use between 4 and 6 inputs while having a maximum of 2 outputs. 
This means that an output variable can depend on up to 6 input variables.
If an application requires more input variables, multiple LUT primitives need to be cascaded. 
Another approach would be to offer LUT primitives with a higher number of inputs (e.g. an 8:1 LUT). This comes at the cost of higher area consumption which is wasted for simpler functions using less than 8 inputs.
Research in FPGA chip design suggests that 4 to 6 inputs is a good balance.
Xilinx LUTs for example can either be configured as a 6:1 LUT or a 5:2 LUT.

The output of a LUT can be registered with a D-type flip-flop stage.
Or the output can be routed to another LUT for additional functionality.
The output register therefore is a configuration option in the bitstream.



		\subsubsection{Fast Carry Chain}

LUTs only offer a low input signal count and can rarely add longer than 2x2 bit numbers.
This is hardly practical which is why longer additions require multiple LUT primitives.
The way an addition works is by starting at the LSB position.
If both numbers are a "1" bit, a carry bit is added to the next higher bit.
Each additional bit in the input vector length therefore translates to an additional LUT cascaded (in series).
This increases the critical path significantly to the point where timing quickly becomes an issue in high speed FPGAs.
To circumvent this problem, multiple versions of a (fast) carry chain were developed and are standard in modern CLB slices.

	

%\include{sections/Architecture}
	


\clearpage
\printglossary[nonumberlist]
\clearpage
%\printglossary[title=Abkürzungsverzeichnis, type=\acronymtype]

\clearpage
\section{References}
\label{sec:lit}
%%---BIBLIOGRAPHY------------------------------------------------------------------------
{\sloppypar
% \printbibliography[heading=bibintoc]
\selectlanguage{english}				%ngerman or english
\printbibliography
}

\pagebreak

%%---List of Figures------------------------------------------------------------------------
%\clearpage
%\listoffigures
%\listoftables

%%---APPENDIX----------------------------------------------------------------------------
\clearpage
%\include{sections/appendix}

%%---NOTES for DEBUG---------------------------------------------------------------------
%\ifdraft{%Do this only if mode=draft
%%requires \usepackage{todonotes})

%\newpage
%\listoftodos[\section{Todo-Notes}]
%\clearpage

%}
{%Do this only if mode=final
}

\end{document}
