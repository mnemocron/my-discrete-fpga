\subsubsection{Fast Carry Chain}

LUTs only offer a low input signal count and can rarely add longer than 2x2 bit numbers.
This is hardly practical which is why longer additions require multiple LUT primitives.
The way an addition works is by starting at the LSB position.
If both numbers are a "1" bit, a carry bit is added to the next higher bit.
Each additional bit in the input vector length therefore translates to an additional LUT cascaded (in series).
This increases the critical path significantly to the point where timing quickly becomes an issue in high speed FPGAs.
To circumvent this problem, multiple versions of a (fast) carry chain were developed and are standard in modern CLB slices.
