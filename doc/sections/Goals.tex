\subsection{The End Goal of this Project}

With the expected cost being way beyond 1'000 USD all while expecting a performance not even close to 20 year old microcontrollers, it is hard to justify this project.
I admit that designing such a machinery will not deliver any reasonable performance let alone create any innovation.
Yet, it will teach me (and hopefully you) about the inner workings of an FPGA and showcase all the difficulties that designers of modern FPGA chips face.

What I aim to achieve is to build an FPGA with enough complexity to run a few relevant circuits. Thereby teaching myself the basics of FPGA architecture while maintaining a functional, reconfigurable hardware. 
The entire FPGA should be able to run basic circuits such as a 4-bit counter or a 4-bit BCD to 7-segment decoder.
Once the project grows further, more specialized logic can be added including block RAM or even DSP slices.

Since I do not plan to open another Pandoras box of synthesis and routing software algorithms, the bitstream is required to be manually managed resulting in a size constraint of around <4kB.

